\documentclass[11pt,a4paper,titlepage,dvipdfmx]{jarticle}
\usepackage{geometry}
\geometry{left=25truemm,right=25truemm,top=25truemm,bottom=37truemm}
\usepackage{multirow}
\usepackage{diagbox}
\usepackage[dvipdfmx]{graphicx}
\usepackage{float}
\usepackage{amsmath,amssymb}
\usepackage{listings,jlisting}
\usepackage{url}
\renewcommand{\lstlistingname}{プログラムリスト}
\renewcommand{\labelenumii}{\arabic{enumii}).}
\lstset{
  basicstyle={\ttfamily},
  identifierstyle={\small},
  commentstyle={\smallitshape},
  keywordstyle={\small\bfseries},
  ndkeywordstyle={\small},
  stringstyle={\small\ttfamily},
  frame={tb},
  breaklines=true,
  columns=[l]{fullflexible},
  numbers=left,
  xrightmargin=0zw,
  xleftmargin=3zw,
  numberstyle={\scriptsize},
  stepnumber=1,
  numbersep=1zw,
  lineskip=-0.5ex
}
\begin{document}
\title{\huge{画像処理}}}
\author{電子情報工学科5年 \\学籍番号:17404}
\date{提出日:2021年12月20日}

\begin{document}
  \maketitle

  \section{目的}
    本科目達成度の確認のため課題を提出する。
  \section{課題1}
    本章では課題1について課題内容、結果およびプログラムを示す。
    \subsection{課題内容}
      次に課題内容を示す.
      \begin{itembox}[l]{課題1}
        画像処理のデータ変換において,次の変換の入力と出力の関係を表す各関数をグラフで表せ.
        \begin{enumerate}
          \item 輝度調整
          \item 等輝度線
        \end{enumerate}
      \end{itembox}


\end{document}